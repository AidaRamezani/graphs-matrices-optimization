\documentclass[11pt]{article}
%%%%%%%%% options for the file macros.tex
\def\showauthornotes{1}
\def\showkeys{0}
\def\showdraftbox{0}
% \allowdisplaybreaks[1]

%% Shamelessly adapted from a scribe template by Sanjeev Arora

%%%%%%%%%%%%%% Packages
% \usepackage[active,tightpage]{preview}
% \renewcommand{\PreviewBorder}{1in}
\usepackage{hyperref}
\usepackage{amsmath,amssymb,amsthm,amstext,amsfonts,bbm,algorithm,algorithmicx,xspace,nicefrac,
  algpseudocode}
\usepackage{color,stmaryrd,enumerate,latexsym,bm,amsfonts,
  subfigure,wrapfig,verbatim,tabularx,textcomp}
\usepackage[small]{caption}
\usepackage{comment} 
\usepackage{epsfig} 
\usepackage{latexsym,nicefrac,bbm}
\usepackage{xspace}
\usepackage{color,fancybox,graphicx,url,subfigure}
\usepackage{enumitem, fullpage}
\usepackage{booktabs}
\usepackage{commath}
\usepackage{mdframed}
\usepackage{pdfsync}
\usepackage{tikz}
\usetikzlibrary {positioning}

%%%%%%%%%%%%%% Use for definitions
\newcommand{\defeq}{\stackrel{\textup{def}}{=}}

%%%%%%%%%%%%%% Theorem Environments
\newtheorem{theorem}{Theorem}[section]
\newtheorem{problem}[theorem]{Problem}
\newtheorem{lemma}[theorem]{Lemma}
\newtheorem{definition}[theorem]{Definition}
\newtheorem{corollary}[theorem]{Corollary}
\newtheorem{conjecture}[theorem]{Conjecture}
\newtheorem{proposition}[theorem]{Proposition}
\newtheorem{fact}[theorem]{Fact}
\newtheorem{remark}[theorem]{Remark}

%%%%%%%%%%%%%% Probability stuff
\DeclareMathOperator*{\pr}{\bf Pr}
\DeclareMathOperator*{\av}{\mathbbm{E}}
\DeclareMathOperator*{\var}{\bf Var}

%%%%%%%%%%%%%% Matrix stuff
\newcommand{\tr}[1]{\mathop{\mbox{Tr}}\left({#1}\right)}
\newcommand{\diag}[1]{{\bf Diag}\left({#1}\right)}

%% Notation for integers, natural numbers, reals, fractions, sets, cardinalities
%%and so on
\newcommand{\nfrac}[2]{\nicefrac{#1}{#2}}
\def\abs#1{\left| #1 \right|}
\renewcommand{\norm}[1]{\ensuremath{\left\lVert #1 \right\rVert}}

\newcommand{\floor}[1]{\left\lfloor\, {#1}\,\right\rfloor}
\newcommand{\ceil}[1]{\left\lceil\, {#1}\,\right\rceil}

\newcommand{\pair}[1]{\left\langle{#1}\right\rangle} %for inner product

\newcommand\B{\{0,1\}}      % boolean alphabet  use in math mode
\newcommand\bz{\mathbb Z}
\newcommand\nat{\mathbb N}
\newcommand\rea{\mathbb R}
\newcommand\com{\mathbb{C}}
\newcommand\plusminus{\{\pm 1\}}
\newcommand\Bs{\{0,1\}^*}   % B star use in math mode
\newcommand{\ones}{\mathbbm{1}}
\newcommand{\eye}{\mathbbm{I}}



\newcommand{\V}[1]{\mathbf{#1}\ignorespaces}
\renewcommand\AA{\boldsymbol{\mathit{A}}}
\newcommand\LL{\boldsymbol{\mathit{L}}}

% Used to denote bold commands
                                % e.g. vectors, matrices
\DeclareRobustCommand{\fracp}[2]{{#1 \overwithdelims()#2}}
\DeclareRobustCommand{\fracb}[2]{{#1 \overwithdelims[]#2}}
\newcommand{\marginlabel}[1]%
{\mbox{}\marginpar{\it{\raggedleft\hspace{0pt}#1}}}
\newcommand\card[1]{\left| #1 \right|} %cardinality of set S; usage \card{S}
\renewcommand\set[1]{\left\{#1\right\}} %usage \set{1,2,3,,}
\renewcommand\complement{\ensuremath{\mathsf{c}}}
\newcommand\poly{\mbox{poly}}  %usage \poly(n)
\newcommand{\comp}[1]{\overline{#1}}
\newcommand{\smallpair}[1]{\langle{#1}\rangle}
\newcommand{\ol}[1]{\ensuremath{\overline{#1}}\xspace}
\newcommand{\eps}{\epsilon}
\DeclareMathOperator{\vol}{\mathsf{vol}}


%%%%%%%%%%%%%% Mathcal shortcuts
\newcommand\calF{\mathcal{F}}
\newcommand\calP{\mathcal{P}}
\newcommand\calS{\mathcal{S}}
\newcommand\calG{\mathcal{G}}
\newcommand\calH{\mathcal{H}}
\newcommand\calC{\mathcal{C}}
\newcommand\calD{\mathcal{D}}
\newcommand\calI{\mathcal{I}}
\newcommand\calV{\mathcal{V}}
\newcommand\calK{\mathcal{K}}
\newcommand\calN{\mathcal{N}}
\newcommand\calX{\mathcal{X}}
\newcommand\calU{\mathcal{U}}
\newcommand\calE{\mathcal{E}}

%%%%%%%%%%%%%% {{{ authornotes }}}
\definecolor{Mygray}{gray}{0.8}

 \ifcsname ifcommentflag\endcsname\else
  \expandafter\let\csname ifcommentflag\expandafter\endcsname
                  \csname iffalse\endcsname
\fi

\ifnum\showauthornotes=1
\newcommand{\todo}[1]{\colorbox{Mygray}{\color{red}#1}}
\else
\newcommand{\todo}[1]{#1}
\fi

\ifnum\showauthornotes=1
\newcommand{\Authornote}[2]{{\sf\small\color{red}{[#1: #2]}}}
\newcommand{\Authoredit}[2]{{\sf\small\color{red}{[#1]}\color{blue}{#2}}}
\newcommand{\Authorcomment}[2]{{\sf \small\color{gray}{[#1: #2]}}}
\newcommand{\Authorfnote}[2]{\footnote{\color{red}{#1: #2}}}
\newcommand{\Authorfixme}[1]{\Authornote{#1}{\textbf{??}}}
\newcommand{\Authormarginmark}[1]{\marginpar{\textcolor{red}{\fbox{%\Large
#1:!}}}}
\else
\newcommand{\Authornote}[2]{}
\newcommand{\Authoredit}[2]{}
\newcommand{\Authorcomment}[2]{}
\newcommand{\Authorfnote}[2]{}
\newcommand{\Authorfixme}[1]{}
\newcommand{\Authormarginmark}[1]{}
\fi


%%%%%%%%%%%%%% Logical operators
\newcommand\true{\mbox{\sc True}}
\newcommand\false{\mbox{\sc False}}
\def\scand{\mbox{\sc and}}
\def\scor{\mbox{\sc or}}
\def\scnot{\mbox{\sc not}}
\def\scyes{\mbox{\sc yes}}
\def\scno{\mbox{\sc no}}

%% Parantheses
\newcommand{\paren}[1]{\unskip\left({#1}\right)}
\newcommand{\sqparen}[1]{\unskip\left[{#1}\right]}
\newcommand{\curlyparen}[1]{\unskip\left\{{#1}\right\}}
\newcommand{\smallparen}[1]{\unskip({#1})}
\newcommand{\smallsqparen}[1]{\unskip[{#1}]}
\newcommand{\smallcurlyparen}[1]{\unskip\{{#1}\}}

%% short-hands for relational simbols

\newcommand{\from}{:}
\newcommand\xor{\oplus}
\newcommand\bigxor{\bigoplus}
\newcommand{\logred}{\leq_{\log}}
\def\iff{\Leftrightarrow}
\def\implies{\Rightarrow}




%% macros to write pseudo-code

\newlength{\pgmtab}  %  \pgmtab is the width of each tab in the
\setlength{\pgmtab}{1em}  %  program environment
 \newenvironment{program}{\renewcommand{\baselinestretch}{1}%
\begin{tabbing}\hspace{0em}\=\hspace{0em}\=%
\hspace{\pgmtab}\=\hspace{\pgmtab}\=\hspace{\pgmtab}\=\hspace{\pgmtab}\=%
\hspace{\pgmtab}\=\hspace{\pgmtab}\=\hspace{\pgmtab}\=\hspace{\pgmtab}\=%
\+\+\kill}{\end{tabbing}\renewcommand{\baselinestretch}{\intl}}
\newcommand {\BEGIN}{{\bf begin\ }}
\newcommand {\ELSE}{{\bf else\ }}
\newcommand {\IF}{{\bf if\ }}
\newcommand {\FOR}{{\bf for\ }}
\newcommand {\TO}{{\bf to\ }}
\newcommand {\DO}{{\bf do\ }}
\newcommand {\WHILE}{{\bf while\ }}
\newcommand {\ACCEPT}{{\bf accept}}
\newcommand {\REJECT}{\mbox{\bf reject}}
\newcommand {\THEN}{\mbox{\bf then\ }}
\newcommand {\END}{{\bf end}}
\newcommand {\RETURN}{\mbox{\bf return\ }}
\newcommand {\HALT}{\mbox{\bf halt}}
\newcommand {\REPEAT}{\mbox{\bf repeat\ }}
\newcommand {\UNTIL}{\mbox{\bf until\ }}
\newcommand {\TRUE}{\mbox{\bf true\ }}
\newcommand {\FALSE}{\mbox{\bf false\ }}
\newcommand {\FORALL}{\mbox{\bf for all\ }}
\newcommand {\DOWNTO}{\mbox{\bf down to\ }}

% Theorem-type environments
% \theoremstyle{break} 
% \theoremheaderfont{\scshape}
% \theorembodyfont{\slshape}
% \newtheorem{Thm}{Theorem}[section]
% \newtheorem{Lem}[Thm]{Lemma}
% \newtheorem{Cor}[Thm]{Corollary}
% \newtheorem{Prop}[Thm]{Proposition}
% % \theoremstyle{plain} 
% % \theorembodyfont{\rmfamily} 
% \newtheorem{Ex}[Thm]{Exercise}
% \newtheorem{Exa}[Thm]{Example}
% \newtheorem{Rem}[Thm]{Remark}
% % \theorembodyfont{\itshape}
% \newtheorem{Def}[Thm]{Definition}
% \newtheorem{Conj}[Thm]{Conjecture}
% \newtheorem{Obs}[Thm]{Observation}
% \newtheorem{Ques}[Thm]{Question}
%\newenvironment{proof}{\noindent {\sc Proof:}}{$\Box$ \medskip} 
\newenvironment{problems} % Definition of problems
 {\renewcommand{\labelenumi}{\S\theenumi}
	\begin{enumerate}}{\end{enumerate}}


%%%%%%%%%%%%%%%%% Proof Environments

\def\FullBox{\hbox{\vrule width 6pt height 6pt depth 0pt}}
%
%\def\qed{\ifmmode\qquad\FullBox\else{\unskip\nobreak\hfil
%\penalty50\hskip1em\null\nobreak\hfil\FullBox
%\parfillskip=0pt\finalhyphendemerits=0\endgraf}\fi}

\def\qedsketch{\ifmmode\Box\else{\unskip\nobreak\hfil
\penalty50\hskip1em\null\nobreak\hfil$\Box$
\parfillskip=0pt\finalhyphendemerits=0\endgraf}\fi}

%\newenvironment{proof}{\begin{trivlist} \item {\bf Proof:~~}}
 %  {\qed\end{trivlist}}

\newenvironment{proofsketch}{\begin{trivlist} \item {\bf
Proof Sketch:~~}}
  {\qedsketch\end{trivlist}}

\newenvironment{proofof}[1]{\begin{trivlist} \item {\bf Proof
#1:~~}}
  {\qed\end{trivlist}}

\newenvironment{claimproof}{\begin{quotation} \noindent
{\bf Proof of claim:~~}}{\qedsketch\end{quotation}}


%%%%%%%%%%%%%%%%%%%%%%%%%%%%%%%%%%%%%%%%%%%%%%%%%%%%%%%%%%%%%%%%%%%%%%%%%%%
%%%%%%%%%%%%%%%%%%%%%%%%%%%%%%%%%%%%%%%%%%%%%%%%%%%%%%%%%%%%%%%%%%%%%%%%%%%




\newlength{\tpush}
\setlength{\tpush}{2\headheight}
\addtolength{\tpush}{\headsep}

\newcommand{\handout}[5]{
   \noindent
   \begin{center}
   \framebox{ \vbox{ \hbox to \textwidth { {\bf \coursenum\ :\  \coursename} \hfill #5 }
       \vspace{3mm}
       \hbox to \textwidth { {\Large \hfill #2  \hfill} }
       \vspace{1mm}
       \hbox to \textwidth { {\it #3 \hfill #4} }
     }
   }
   \end{center}
   \vspace*{4mm}
   \newcommand{\lecturenum}{#1}
   \addcontentsline{toc}{chapter}{Lecture #1 -- #2}
}

\newcommand{\lecturetitle}[4]{\handout{#1}{#2}{Lecturer: \courseprof
  }{Scribe: #3}{Lecture #1 : #4}}
\newcommand{\guestlecturetitle}[5]{\handout{#1}{#2}{Lecturer:
    #4}{Scribe: #3}{Lecture #1 - #5}}


%%%%%%%%%%%%%%%%%%%%%%%%%%%%%%%%%%%%%%%%%%%%%%%%%%%%%%%%%
%%% Commands to include figures


%% PSfigure

\newcommand{\PSfigure}[3]{\begin{figure}[t] 
  \centerline{\vbox to #2 {\vfil \psfig{figure=#1.eps,height=#2} }} 
  \caption{#3}\label{#1} 
  \end{figure}} 
\newcommand{\twoPSfigures}[5]{\begin{figure*}[t]
  \centerline{%
    \hfil
    \begin{tabular}{c}
        \vbox to #3 {\vfil\psfig{figure=#1.eps,height=#3}} \\ (a)
    \end{tabular}
    \hfil\hfil\hfil
    \begin{tabular}{c}
        \vbox to #3 {\vfil\psfig{figure=#2.eps,height=#3}} \\ (b)
    \end{tabular}
    \hfil}
  \caption{#4}
  \label{#5}
%  \sublabel{#1}{(a)}
%  \sublabel{#2}{(b)}
  \end{figure*}}

\newcounter{fignum}

% fig
%command to insert figure. usage \fig{name}{h}{caption}
%where name.eps is the postscript file and h is the height in inches
%The figure is can be referred to using \ref{name}
\newcommand{\fig}[3]{%
\begin{minipage}{\textwidth}
\centering\epsfig{file=#1.eps,height=#2}
\caption{#3} \label{#1}
\end{minipage}
}%


% ffigure
% Usage: \ffigure{name of file}{height}{caption}{label}
\newcommand{\ffigure}[4]{\begin{figure} 
  \centerline{\vbox to #2 {\hfil \psfig{figure=#1.eps,height=#2} }} 
  \caption{#3}\label{#4} 
  \end{figure}} 

% ffigureh
% Usage: \ffigureh{name of file}{height}{caption}{label}
\newcommand{\ffigureh}[4]{\begin{figure}[!h] 
  \centerline{\vbox to #2 {\vfil \psfig{figure=#1.eps,height=#2} }} 
  \caption{#3}\label{#4} 
  \end{figure}} 


% {{{ draftbox }}}
\ifnum\showdraftbox=1
\newcommand{\draftbox}{\begin{center}
  \fbox{%
    \begin{minipage}{2in}%
      \begin{center}%
%        \begin{Large}%
          \large\textsc{Working Draft}\\%
%        \end{Large}\\
        Please do not distribute%
      \end{center}%
    \end{minipage}%
  }%
\end{center}
\vspace{0.2cm}}
\else
\newcommand{\draftbox}{}
\fi


%% Complexity classes
\newcommand\p{\mbox{\bf P}\xspace}
\newcommand\np{\mbox{\bf NP}\xspace}
\newcommand\cnp{\mbox{\bf coNP}\xspace}
\newcommand\sigmatwo{\mbox{\bf $\Sigma_2$}\xspace}
\newcommand\ppoly{\mbox{\bf $\p_{\bf /poly}$}\xspace}
\newcommand\sigmathree{\mbox{\bf $\Sigma_3$}\xspace}
\newcommand\pitwo{\mbox{\bf $\Pi_2$}\xspace}
\newcommand\rp{\mbox{\bf RP}\xspace}
\newcommand\zpp{\mbox{\bf ZPP}\xspace}
\newcommand\bpp{\mbox{\bf BPP}\xspace}
\newcommand\ph{\mbox{\bf PH}\xspace}
\newcommand\pspace{\mbox{\bf PSPACE}\xspace}
\newcommand\npspace{\mbox{\bf NPSPACE}\xspace}
\newcommand\dl{\mbox{\bf L}\xspace}
\newcommand\ma{\mbox{\bf MA}\xspace}
\newcommand\am{\mbox{\bf AM}\xspace}
\newcommand\nl{\mbox{\bf NL}\xspace}
\newcommand\conl{\mbox{\bf coNL}\xspace}
\newcommand\sharpp{\mbox{\#{\bf P}}\xspace}
\newcommand\parityp{\mbox{$\oplus$ {\bf P}}\xspace}
\newcommand\ip{\mbox{\bf IP}\xspace}
\newcommand\pcp{\mbox{\bf PCP}}
\newcommand\dtime{\mbox{\bf DTIME}}
\newcommand\ntime{\mbox{\bf NTIME}}
\newcommand\dspace{\mbox{\bf SPACE}\xspace}
\newcommand\nspace{\mbox{\bf NSPACE}\xspace}
\newcommand\cnspace{\mbox{\bf coNSPACE}\xspace}
\newcommand\exptime{\mbox{\bf EXP}\xspace}
\newcommand\nexptime{\mbox{\bf NEXP}\xspace}
\newcommand\genclass{\mbox{$\cal C$}\xspace}
\newcommand\cogenclass{\mbox{\bf co$\cal C$}\xspace}
\newcommand\size{\mbox{\bf SIZE}\xspace}
\newcommand\sig{\mathbf \Sigma}
\newcommand\pip{\mathbf \Pi}

%%Computational problems
\newcommand\sat{\mbox{SAT}\xspace}
\newcommand\tsat{\mbox{3SAT}\xspace}
\newcommand\tqbf{\mbox{TQBF}\xspace}

\allowdisplaybreaks

\usepackage{tikz}

\usepackage[
    backend=biber,
% giveninits=true,
% natbib=true,
    style=alphabetic,
    url=false, 
 %   doi=true,
    hyperref,
    backref=true,
    backrefstyle=none,
    maxbibnames=10,
    sortcites
]{biblatex}
\addbibresource{papers.bib}

%%%%%%%%% Authornotes
\newcommand{\Snote}{\Authornote{S}}

\newenvironment{tight_enumerate}{
\begin{enumerate}
 \setlength{\itemsep}{2pt}
 \setlength{\parskip}{1pt}
}{\end{enumerate}}
\newenvironment{tight_itemize}{
\begin{itemize}
 \setlength{\itemsep}{2pt}
 \setlength{\parskip}{1pt}
}{\end{itemize}}



\addbibresource{refs.bib}

%%%%%%%%%%%%%%%%%%%%%%%%%

\begin{document}
\newcommand{\coursenum}{{CSC 2421H}}
\newcommand{\coursename}{{Graphs, Matrices, and Optimization}}
\newcommand{\courseprof}{Sushant Sachdeva}

\lecturetitle{4}{Random Walk}{Junwei Sun}{10 1 2018}
\textbf{HW:} The proof of statement is left as exercise for the student
\section{Remark from last class}
\label{sec:goal}

If at step t, your distribution is given by $\mathbf{p}_{t}$, then the next distribution $\mathbf{p}_{t+1}$ is given by: 
\begin{equation}
    \mathbf{p}_{t+1} = \mathbf{AD}^{-1}\mathbf{p}_{t},
\end{equation}
where:\\
\textbf{A}: weighted adjacency matrix\\
\textbf{D}: diagonal weighted degree matrix\\\\
The state transition can also be expressed as:
\begin{equation*}
    \mathbf{p}_{t+1}(x) = \sum_{y:(x,y) \in E} \frac{w(x,y)}{\mathbf{D}(y)}*\mathbf{p}_{t}(y)
\end{equation*}

\section{Stationary State}
\begin{definition}
    If distribution $\mathbf{\pi \in \rea^v}$ is said to be stationary distribution for G if $\mathbf{AD}^{-1}\pi = \pi$ 
\end{definition}

\begin{lemma}
Any undirected graph has a stationary distribution
\end{lemma}

\begin{proof}
Given any undirected graph G, let
\begin{equation*}
    \pi = \frac{1}{\mathbf{1D^\top1}}\mathbf{D1}
\end{equation*}
% $$\pi = \frac{1}{\mathbf{1D^\top1}}\mathbf{D1}$$.
This is the stationary distribution for G since
\begin{equation*}
    \mathbf{AD}^{-1}\pi = \frac{1}{\mathbf{1^\top D1}}\mathbf{A1} = \frac{1}{\mathbf{1^\top D1}}\mathbf{D1 = \pi}
\end{equation*}
% $$\mathbf{AD}^{-1}\pi = \frac{1}{\mathbf{1^\top D1}}\mathbf{A1} = \frac{1}{\mathbf{1^\top D1}}\mathbf{D1 = \pi}$$
\end{proof}
\textbf{Claim:}
If G is connected, $\pi$ is unique\\\\
\textbf{Remark:}
Even if G is connected, it is not true that for any $\mathbf{p_0}$, $\mathbf{p}_t \rightarrow \pi$\\

\newpage

\textbf{Example:}\\
\begin {tikzpicture}[-latex ,auto ,node distance =4 cm and 5cm ,on grid ,
semithick ,
state/.style ={ circle ,top color =white,
draw, text=blue , minimum width =1 cm}]
\node[state] (C){$1$};
\node[state] (A) [left=of C] {$0$};
\path (C) edge [bend left =25] node[below =0.15 cm] {$\frac{1}{2}$} (A);
\path (A) edge [bend right = -15] node[below =0.15 cm] {$\frac{1}{2}$} (C);
\end{tikzpicture}

The stationary state $\pi$ = ($\frac{1}{2}$,$\frac{1}{2}$) but the random walk will alternate between Vertex 0 and Vertex 1\\

\section{Positive Semi-Definite(PSD)}

\begin{definition}
    A symmetric matrix \textbf{M} is positive semi-definite(psd) if $\forall$ \textbf{x}, $\mathbf{x^\top Mx \geq}$ 0
\end{definition}

\begin{theorem}
     the following statements are equivalent:\\
(1) \textbf{M} is psd\\
(2) All eigenvalues of \textbf{M} are non-negative\\
(3) There exist an matrix \textbf{A} such that $\mathbf{M = AA^\top}$\\
\end{theorem}

\begin{lemma}
    If \textbf{M} is psd, then for all matrices \textbf{C}, $\mathbf{C^\top}$\textbf{MC} is psd\\
\end{lemma} 
\begin{proof}
$\forall$ \textbf{x}, $\mathbf{x^\top C^\top MCx = (Cx)^\top M(Cx) \geq}$ 0 since \textbf{M} is psd\\
\end{proof}
\textbf{Notation:} \textbf{M} is psd $\Leftrightarrow \mathbf{M \succeq}$ 0
\begin{lemma}
    $\mathbf{\mathcal{L}} \succeq$ 0 where $\mathbf{\mathcal{L}}$ is the laplacian matrix of some graph
\end{lemma}
\textbf{Remark:} $\mathbf{\mathcal{L}} \succeq$ 0 implies $\mathbf{N} \succeq$ 0 since \textbf{N} = $\mathbf{D^{-\frac{1}{2}}\mathcal{L}D^{-\frac{1}{2}}}$
\begin{lemma}
    $\mathbf{\mathcal{L}} \preccurlyeq$ 2\textbf{D} $\Leftrightarrow$ \textbf{N} $\preccurlyeq$ 2\textbf{I} $\Leftrightarrow \lambda_i(\mathbf{N}), \nu_i \leq 2$ 
\end{lemma} 
\textbf{HW:} If \textbf{A} $\succeq$ \textbf{B}, then $\lambda_i(\mathbf{A}) \geq \lambda_i(\mathbf{B})$\\

\section{Lazy random walk}
\subsection{Lazy random walk matrix}
At each step, the lazy random walk will do the following
\[
\begin{cases} 
      $with probability$ \frac{1}{2} & $stay at the current vertex$ \\
      $with probability$ \frac{1}{2} & $take a usual random step$ \\
\end{cases}
\]
Lazy Random Walk Transition Matrix \textbf{W =  $\frac{1}{2}\mathbf{(I + AD^{-1})}$}\\
We know that the normalized Laplacian(\textbf{N}) can be expressed as:\\
\begin{equation*}
\begin{split}
 \mathbf{N} & = \mathbf{D}^{-\frac{1}{2}}\mathbf{\mathcal{L}D}^{-\frac{1}{2}}\\
   & = \mathbf{D}^{-\frac{1}{2}}\mathbf{(D-A)D}^{-\frac{1}{2}}\\
   & = \mathbf{I - D}^{-\frac{1}{2}}\mathbf{AD}^{-\frac{1}{2}}
\end{split}    
\end{equation*}
applied this result to the lazy walk transition matrix\\
\begin{equation*}
\begin{split}
 \mathbf{W} & =  \frac{1}{2}\mathbf{I} + \frac{1}{2}\mathbf{AD}^{-1} \\
   & = \frac{1}{2}\mathbf{I} + \frac{1}{2}\mathbf{D}^{\frac{1}{2}}\mathbf{D}^{-\frac{1}{2}}\mathbf{AD}^{-\frac{1}{2}}\mathbf{D}^{-\frac{1}{2}}\\
   & = \frac{1}{2}\mathbf{I} + \frac{1}{2}\mathbf{D}^{\frac{1}{2}}\mathbf{(I-N)D}^{-\frac{1}{2}}\\
   & = \mathbf{I} - \frac{1}{2}\mathbf{D}^{\frac{1}{2}}\mathbf{ND}^{-\frac{1}{2}}
\end{split}    
\end{equation*}

Thus,we can express the lazy random walk transition matrix as:\\
\begin{equation}
    \mathbf{W = I} - \frac{1}{2}\mathbf{D}^{\frac{1}{2}}\mathbf{ND}^{-\frac{1}{2}}
\end{equation}

\subsection{Eigenpair for lazy random walk matrix}

\begin{lemma}
    If$( \nu_i,\psi_i)$ is an eigenpair for \textbf{N}, i.e \textbf{N}$\psi_i$ = $ \nu_i\psi_i \Leftrightarrow$ $(1-\frac{1}{2} \nu_i,D^{\frac{1}{2}}\psi_i)$ is an eigenpair for \textbf{W}\\
\end{lemma}
\begin{proof}
\begin{equation*}
    \begin{split}
     \mathbf{WD}^{\frac{1}{2}}\psi_i       
     &=\mathbf{(I}-\frac{1}{2}\mathbf{D}^{\frac{1}{2}}\mathbf{ND}^{-\frac{1}{2}})\mathbf{D}^{\frac{1}{2}}\psi_i\\
     &=\mathbf{D}^{\frac{1}{2}}\psi_i - \frac{1}{2} \mathbf{\nu_iD}^{\frac{1}{2}}\psi_i
    \end{split}
\end{equation*}
\end{proof}

Because of lamma 4.1 and lemma 3.5, we can obtain the following corollary\\
\textbf{corollary:} 0$\leq \lambda_i(\textbf{W}) \leq$ 1\\
\textbf{Warning:} W is not symmetric. Thus, its eigenvector need not be orthogonal\\

\section{Convergence of Lazy Random Walk}
\subsection{Finding an expression for $\mathbf{p}_t$}
State transition from $\mathbf{p}_t$ to $\mathbf{p}_{t+1}$ in a lazy random is given by :\\
$$\mathbf{p}_{t+1} = \mathbf{Wp}_t$$\\
When t = 0:
$$\mathbf{p}_1 = \mathbf{Wp}_0$$\\
We know that\\
$$\mathbf{D}^{-\frac{1}{2}}\mathbf{p}_0 = \sum_{i=1}^{n}\alpha_i\psi_i \Leftrightarrow \mathbf{p}_0 = \sum_{i=1}^{n}\alpha_i\mathbf{D}^{\frac{1}{2}}\psi_i$$\\
Thus, we can express $\mathbf{p}_1$ as:\\
$$\mathbf{p}_1 = \mathbf{Wp}_0 = \sum_{i=1}^{n}\alpha_i(\mathbf{WD}^{\frac{1}{2}}\psi_i) = \sum_{i=1}^{n}\alpha_i(1-\frac{\nu_i}{2})\mathbf{D}^{\frac{1}{2}}\psi_i$$\\
Iterating the process above, we obtain:\\
$$\mathbf{p}_t = \sum_{i=1}^{n}\alpha_i(1-\frac{\nu_i}{2})^\top \mathbf{D}^{\frac{1}{2}}\psi_i$$\\

\textbf{Claim:} If G is connected $\Leftrightarrow \nu_2 >$ 0

\textbf{Remark:} the claim above implies the following:\\
$$\forall i \neq 1, 0\leq 1 - \frac{\nu_i}{2} < 1$$\\

\subsection{Given $\epsilon$, finding step t such that $\mathbf{p}_t$ is $\epsilon$ closed to the stationary distribution}
At arbitrary vertex u, we have the following:
\begin{equation}
\begin{split}
 \mathbf{1_u^\top P_t - 1_u^\top \pi }& \mathbf{= 1_u^\top P_t - \frac{1^\top _uD1}{1^\top D1}}\\
         &= \sum_{i=1}^{n}\alpha_i(1-\frac{\nu_i}{2})^\top \mathbf{1_u^\top D^{\frac{1}{2}}\psi_i-\frac{1^\top _uD1}{1^\top D1}}
\end{split}
\end{equation}
We know that:\\
\begin{equation*}
    \begin{split}
         \psi_1 = \frac{(\mathbf{D}^{\frac{1}{2}}\mathbf{1})}{||\mathbf{D}^{\frac{1}{2}}\mathbf{1}||} = \frac{(\mathbf{D}^{\frac{1}{2}}\mathbf{1})}{\sqrt{\mathbf{1}^\top \mathbf{D1}}}
    \end{split}
\end{equation*}
multiplied both side with $\mathbf{1_u^\top D}^{\frac{1}{2}}$, we get
\begin{equation}
    \begin{split}
         \mathbf{1_u^\top D}^{\frac{1}{2}}\psi_1 
         &\mathbf{= 1_u^\top D}^{\frac{1}{2}}\frac{(\mathbf{D}^{\frac{1}{2}}\mathbf{1})}{||\mathbf{D}^{\frac{1}{2}}\mathbf{1||} }\\
         &\mathbf{= \frac{(1_u^\top D1)}{\sqrt{1^\top D1}}}
    \end{split}
\end{equation}
We can express $\mathbf{D}^{-\frac{1}{2}}\mathbf{p}_0$ as following
\begin{equation*}
    \begin{split}
         \mathbf{D}^{-\frac{1}{2}}\mathbf{p}_0 &\mathbf{= \sum_{i=1}^{n}\alpha_i\psi_i}\\
    \end{split}
\end{equation*}
multiply each side with $\psi^\top _1$\\
\begin{equation*}
    \begin{split}
         \psi^\top _1\mathbf{D}^{-\frac{1}{2}}\mathbf{p}_0 &\mathbf{= \alpha_1} \\
    \end{split}
\end{equation*}
This gives us:
\begin{equation*}
    \begin{split}
        \alpha_1 &= \frac{\mathbf{(1^\top D}^{\frac{1}{2}})\mathbf{D}^{-\frac{1}{2}}\mathbf{p}_0}{\sqrt{\mathbf{1^\top D1}}} \\
    \end{split}
\end{equation*}
because $\mathbf{p}_0$ is a probability vector and sum up to 1, we have,
\begin{equation}
    \begin{split}
          &= \frac{1}{\mathbf{\sqrt{1^\top D1}}}
    \end{split}
\end{equation}

Now, we can further simplify (3) as:
\begin{equation*}
    \begin{split}
         \mathbf{1^\top _up_t - 1_u^\top \pi}
         &= \alpha_1(1-\frac{\nu_1}{2})^\top \psi_1 +\sum_{i=2}^{n}\alpha_i(1-\frac{\nu_i}{2})^\top \mathbf{1_u^\top D}^{\frac{1}{2}}\psi_i -\mathbf{\frac{1^\top _uD1}{1^\top D1}}\\
        %  $with equation(4) and (5) we obtained$\\
         &= \sum_{i=2}^{n}\alpha_i(1-\frac{\nu_i}{2})^\top 1_u^\top \mathbf{D}^{\frac{1}{2}}\psi_i
    \end{split}
\end{equation*}
Combining the result above, we have the following
\begin{equation}
    \begin{split}
        \mathbf{|1_u^\top p}_t - \mathbf{1_u^\top \pi|} 
        &\leq \sum_{i=2}^{n}|\alpha_i\mathbf{1_u^\top D}^{\frac{1}{2}}\psi_i|(1-\frac{\nu_i}{2})^\top\\
        % $because \nu_2 \leq \nu_3 \leq ...$,we have \\
        &\leq (1-\frac{\nu_2}{2})^\top \sum_{i=2}^{n}|\alpha_i\mathbf{1_u^\top D}^{\frac{1}{2}}\psi_i|\\
        &\leq (1-\frac{\nu_2}{2})^\top \sqrt{\sum_{i=2}^{n}\alpha_i^2\sum_{i=2}^{n}(\mathbf{1_u^\top D}^{\frac{1}{2}}\psi_i)^2}
    \end{split}
\end{equation}
Now let's try to simplify the term inside the square root, starting with $\sum_{i=2}^{n}\alpha_i^2$
\begin{equation}
    \begin{split}
         \sum_{i=2}^{n}\alpha_i^2 
         & \mathbf{\leq ||D}^{-\frac{1}{2}}\mathbf{p}_0||_2^2 \\
        %  $Assuming we start at vertex v$\\
         &\mathbf{\leq 1_v^\top D}^{-1}\mathbf{1_v}\\ 
         &= \frac{1}{D(v)}
    \end{split}
\end{equation}
Now let's simplify $\sum_{i=2}^{n}(1_u^\top D^{\frac{1}{2}}\psi_i)^2$\\
We shall start with finding an expression for $\mathbf{D}^{\frac{1}{2}}\mathbf{1_u}$\\\\
\textbf{Claim:} 
\begin{equation*}
    \mathbf{D}^{\frac{1}{2}}\mathbf{1_u} = \sum_{i=1}^{n}(\mathbf{1_u^\top D}^{\frac{1}{2}}\psi_i)\psi_j
\end{equation*}


\begin{proof}
Let's express $\mathbf{D}^{\frac{1}{2}}\mathbf{1_u}$ with eigenvector and eigenvalue
\begin{equation}
    \begin{split}
         \mathbf{D}^{\frac{1}{2}}\mathbf{1_u}
         &\mathbf{= \sum_{i=1}^{n} \beta_i\psi_i}\\
        %  $multiplied both sides with \psi_j$, we obtain\\
         \mathbf{\psi_j^\top D}^{\frac{1}{2}}\mathbf{1_u}
         &= \mathbf{\beta_j}\\
        %  $take the transpose of left hand side$\\
         \mathbf{(\psi_j^\top D}^{\frac{1}{2}}\mathbf{1_u)^\top } 
         &\mathbf{= 1_u^\top D}^{\frac{1}{2}}\psi_j \\
         \\
         \mathbf{||D}^{\frac{1}{2}}\mathbf{1_u}||^2_2 
         &\mathbf{= (D}^{\frac{1}{2}}\mathbf{1_u)^\top (D}^{\frac{1}{2}}\mathbf{1_u)}\\
         &=(\sum_{i=1}^{n}\beta_i\psi_i)^\top (\sum_{j=1}^{n}\beta_j\psi_j)\\
         &=\sum_{i=1}^{n}\beta_i^2 \\
         &= \sum_{i=1}^{n}(\mathbf{1_u^\top D}^{\frac{1}{2}}\psi_i)^2
    \end{split}
\end{equation}    
\end{proof}
With the results above, (6) can be further simplified as following:
\begin{equation}
    \begin{split}
        \mathbf{|1_u^\top p_t - 1_u^\top \pi|} 
        &\leq (1-\nu_2)^\top \sqrt{||\mathbf{D}^{-\frac{1}{2}}\mathbf{p}_0||^2_2||\mathbf{D}^{\frac{1}{2}}\mathbf{1_u}||^2_2}\\
        &\mathbf{=||D}^{\frac{1}{2}}\mathbf{1_u||\cdot||D}^{-\frac{1}{2}}\mathbf{p}_0||(1-\frac{\nu_2}{2})^\top \\
        % $with (7) and assuming the walk starts at v, then$\\
        &=\sqrt{\frac{D_u}{D_v}}(1-\frac{\nu_2}{2})^\top \\
        % $Notice that$ \mathbf{(1-x) \leq e^{-x}}\\
        &\leq \sqrt{\frac{D_u}{D_v}}e^{-\nu_2t/2}
    \end{split}
\end{equation}

\begin{theorem}
Given an undirected unweighted graph G and $\epsilon$, it suffices for the lazy random walk to take t steps to get $\epsilon$ closed to stationary distribution.\\
In other word, if\\
$$t \geq \frac{2}{\nu_2}\log(\frac{n}{\epsilon})$$ 
then 
$$\mathbf{|1_u^\top p}_t - \mathbf{1_u^\top }\pi| \leq \epsilon$$
\end{theorem}

\textbf{Interpretation:}
On a graph with n vertex, if the lazy random walk want to be $\frac{\epsilon}{2}$ closed to the stationary distribution, then it would satisfy the following relation.\\
\begin{equation*}
    \begin{split}
       T(\frac{\epsilon}{2})
        &= \frac{2}{\nu_2}log(\frac{2n}{\epsilon})\\
        &=\Theta(\frac{1}{\nu_2}) + T(\epsilon)
    \end{split}
\end{equation*}

\subsection{Application of the theorem on some examples}
$K_n$:complete graph with n vertices\\
Lazy random walk mixes in $\Omega(\log n)$ steps\\
\begin{equation*}
    \begin{array}{cc}
        \mathcal{L}_{K_n} = n\textbf{I} - \mathbf{11^\top}\\
        \lambda_i(L) = 
        \begin{cases} 
        0 & i=1 \\
        n & o/w \\
        \end{cases}\\
        \nu_2(N_{K_n}) =  1
    \end{array}
\end{equation*}
the theorem gives that the lazy random walk would mix in $O(\log n)$. In this case, the bound given by the theorem is tight\\

$R_n$:n-ring graph\\
The second eigenvalue of n-ring graph is given by:\\
$$\nu_2(N_{K_n}) = \theta(\frac{1}{n^2})$$
thus, the theorem gives that the lazy random walk mixes in $O(n^2\log
n)$ steps.\\
the lazy random walk on the n-ring graph can be defined as following:\\
\begin{equation*}
    \begin{array}{cc}
        x_{i+1} =
        \begin{cases} 
        x_{i} & $with probability$ \frac{1}{2} \\
        x_{i}+1 & $with probability$ \frac{1}{4} \\
        x_{i}-1 & $with probability$ \frac{1}{4}\\
        \end{cases}\\
    \end{array}
\end{equation*}
        $$E[x_{i+1}|x_{i}] = x_i$$
        $$E[x_{i+1}^2|x_i] = x_i^2 + \frac{1}{2}$$\\
thus, the Lazy Random Walk is mixed in $\Omega(n^2)$ steps. In this case, the bound given by the theorem is off up to $\log$n


\printbibliography
\end{document}


%%% Local Variables:
%%% mode: latex
%%% TeX-master: t
%%% End:
