\documentclass[11pt]{article}
%%%%%%%%% options for the file macros.tex

\def\showauthornotes{1}
\def\showkeys{0}
\def\showdraftbox{0}
% \allowdisplaybreaks[1]

%% Shamelessly adapted from a scribe template by Sanjeev Arora

%%%%%%%%%%%%%% Packages
% \usepackage[active,tightpage]{preview}
% \renewcommand{\PreviewBorder}{1in}
\usepackage{hyperref}
\usepackage{amsmath,amssymb,amsthm,amstext,amsfonts,bbm,algorithm,algorithmicx,xspace,nicefrac,
  algpseudocode}
\usepackage{color,stmaryrd,enumerate,latexsym,bm,amsfonts,
  subfigure,wrapfig,verbatim,tabularx,textcomp}
\usepackage[small]{caption}
\usepackage{comment} 
\usepackage{epsfig} 
\usepackage{latexsym,nicefrac,bbm}
\usepackage{xspace}
\usepackage{color,fancybox,graphicx,url,subfigure}
\usepackage{enumitem, fullpage}
\usepackage{booktabs}
\usepackage{commath}
\usepackage{mdframed}
\usepackage{pdfsync}
\usepackage{tikz}
\usetikzlibrary {positioning}

%%%%%%%%%%%%%% Use for definitions
\newcommand{\defeq}{\stackrel{\textup{def}}{=}}

%%%%%%%%%%%%%% Theorem Environments
\newtheorem{theorem}{Theorem}[section]
\newtheorem{problem}[theorem]{Problem}
\newtheorem{lemma}[theorem]{Lemma}
\newtheorem{definition}[theorem]{Definition}
\newtheorem{corollary}[theorem]{Corollary}
\newtheorem{conjecture}[theorem]{Conjecture}
\newtheorem{proposition}[theorem]{Proposition}
\newtheorem{fact}[theorem]{Fact}
\newtheorem{remark}[theorem]{Remark}

%%%%%%%%%%%%%% Probability stuff
\DeclareMathOperator*{\pr}{\bf Pr}
\DeclareMathOperator*{\av}{\mathbbm{E}}
\DeclareMathOperator*{\var}{\bf Var}

%%%%%%%%%%%%%% Matrix stuff
\newcommand{\tr}[1]{\mathop{\mbox{Tr}}\left({#1}\right)}
\newcommand{\diag}[1]{{\bf Diag}\left({#1}\right)}

%% Notation for integers, natural numbers, reals, fractions, sets, cardinalities
%%and so on
\newcommand{\nfrac}[2]{\nicefrac{#1}{#2}}
\def\abs#1{\left| #1 \right|}
\renewcommand{\norm}[1]{\ensuremath{\left\lVert #1 \right\rVert}}

\newcommand{\floor}[1]{\left\lfloor\, {#1}\,\right\rfloor}
\newcommand{\ceil}[1]{\left\lceil\, {#1}\,\right\rceil}

\newcommand{\pair}[1]{\left\langle{#1}\right\rangle} %for inner product

\newcommand\B{\{0,1\}}      % boolean alphabet  use in math mode
\newcommand\bz{\mathbb Z}
\newcommand\nat{\mathbb N}
\newcommand\rea{\mathbb R}
\newcommand\com{\mathbb{C}}
\newcommand\plusminus{\{\pm 1\}}
\newcommand\Bs{\{0,1\}^*}   % B star use in math mode
\newcommand{\ones}{\mathbbm{1}}
\newcommand{\eye}{\mathbbm{I}}



\newcommand{\V}[1]{\mathbf{#1}\ignorespaces}
\renewcommand\AA{\boldsymbol{\mathit{A}}}
\newcommand\LL{\boldsymbol{\mathit{L}}}

% Used to denote bold commands
                                % e.g. vectors, matrices
\DeclareRobustCommand{\fracp}[2]{{#1 \overwithdelims()#2}}
\DeclareRobustCommand{\fracb}[2]{{#1 \overwithdelims[]#2}}
\newcommand{\marginlabel}[1]%
{\mbox{}\marginpar{\it{\raggedleft\hspace{0pt}#1}}}
\newcommand\card[1]{\left| #1 \right|} %cardinality of set S; usage \card{S}
\renewcommand\set[1]{\left\{#1\right\}} %usage \set{1,2,3,,}
\renewcommand\complement{\ensuremath{\mathsf{c}}}
\newcommand\poly{\mbox{poly}}  %usage \poly(n)
\newcommand{\comp}[1]{\overline{#1}}
\newcommand{\smallpair}[1]{\langle{#1}\rangle}
\newcommand{\ol}[1]{\ensuremath{\overline{#1}}\xspace}
\newcommand{\eps}{\epsilon}
\DeclareMathOperator{\vol}{\mathsf{vol}}


%%%%%%%%%%%%%% Mathcal shortcuts
\newcommand\calF{\mathcal{F}}
\newcommand\calP{\mathcal{P}}
\newcommand\calS{\mathcal{S}}
\newcommand\calG{\mathcal{G}}
\newcommand\calH{\mathcal{H}}
\newcommand\calC{\mathcal{C}}
\newcommand\calD{\mathcal{D}}
\newcommand\calI{\mathcal{I}}
\newcommand\calV{\mathcal{V}}
\newcommand\calK{\mathcal{K}}
\newcommand\calN{\mathcal{N}}
\newcommand\calX{\mathcal{X}}
\newcommand\calU{\mathcal{U}}
\newcommand\calE{\mathcal{E}}

%%%%%%%%%%%%%% {{{ authornotes }}}
\definecolor{Mygray}{gray}{0.8}

 \ifcsname ifcommentflag\endcsname\else
  \expandafter\let\csname ifcommentflag\expandafter\endcsname
                  \csname iffalse\endcsname
\fi

\ifnum\showauthornotes=1
\newcommand{\todo}[1]{\colorbox{Mygray}{\color{red}#1}}
\else
\newcommand{\todo}[1]{#1}
\fi

\ifnum\showauthornotes=1
\newcommand{\Authornote}[2]{{\sf\small\color{red}{[#1: #2]}}}
\newcommand{\Authoredit}[2]{{\sf\small\color{red}{[#1]}\color{blue}{#2}}}
\newcommand{\Authorcomment}[2]{{\sf \small\color{gray}{[#1: #2]}}}
\newcommand{\Authorfnote}[2]{\footnote{\color{red}{#1: #2}}}
\newcommand{\Authorfixme}[1]{\Authornote{#1}{\textbf{??}}}
\newcommand{\Authormarginmark}[1]{\marginpar{\textcolor{red}{\fbox{%\Large
#1:!}}}}
\else
\newcommand{\Authornote}[2]{}
\newcommand{\Authoredit}[2]{}
\newcommand{\Authorcomment}[2]{}
\newcommand{\Authorfnote}[2]{}
\newcommand{\Authorfixme}[1]{}
\newcommand{\Authormarginmark}[1]{}
\fi


%%%%%%%%%%%%%% Logical operators
\newcommand\true{\mbox{\sc True}}
\newcommand\false{\mbox{\sc False}}
\def\scand{\mbox{\sc and}}
\def\scor{\mbox{\sc or}}
\def\scnot{\mbox{\sc not}}
\def\scyes{\mbox{\sc yes}}
\def\scno{\mbox{\sc no}}

%% Parantheses
\newcommand{\paren}[1]{\unskip\left({#1}\right)}
\newcommand{\sqparen}[1]{\unskip\left[{#1}\right]}
\newcommand{\curlyparen}[1]{\unskip\left\{{#1}\right\}}
\newcommand{\smallparen}[1]{\unskip({#1})}
\newcommand{\smallsqparen}[1]{\unskip[{#1}]}
\newcommand{\smallcurlyparen}[1]{\unskip\{{#1}\}}

%% short-hands for relational simbols

\newcommand{\from}{:}
\newcommand\xor{\oplus}
\newcommand\bigxor{\bigoplus}
\newcommand{\logred}{\leq_{\log}}
\def\iff{\Leftrightarrow}
\def\implies{\Rightarrow}




%% macros to write pseudo-code

\newlength{\pgmtab}  %  \pgmtab is the width of each tab in the
\setlength{\pgmtab}{1em}  %  program environment
 \newenvironment{program}{\renewcommand{\baselinestretch}{1}%
\begin{tabbing}\hspace{0em}\=\hspace{0em}\=%
\hspace{\pgmtab}\=\hspace{\pgmtab}\=\hspace{\pgmtab}\=\hspace{\pgmtab}\=%
\hspace{\pgmtab}\=\hspace{\pgmtab}\=\hspace{\pgmtab}\=\hspace{\pgmtab}\=%
\+\+\kill}{\end{tabbing}\renewcommand{\baselinestretch}{\intl}}
\newcommand {\BEGIN}{{\bf begin\ }}
\newcommand {\ELSE}{{\bf else\ }}
\newcommand {\IF}{{\bf if\ }}
\newcommand {\FOR}{{\bf for\ }}
\newcommand {\TO}{{\bf to\ }}
\newcommand {\DO}{{\bf do\ }}
\newcommand {\WHILE}{{\bf while\ }}
\newcommand {\ACCEPT}{{\bf accept}}
\newcommand {\REJECT}{\mbox{\bf reject}}
\newcommand {\THEN}{\mbox{\bf then\ }}
\newcommand {\END}{{\bf end}}
\newcommand {\RETURN}{\mbox{\bf return\ }}
\newcommand {\HALT}{\mbox{\bf halt}}
\newcommand {\REPEAT}{\mbox{\bf repeat\ }}
\newcommand {\UNTIL}{\mbox{\bf until\ }}
\newcommand {\TRUE}{\mbox{\bf true\ }}
\newcommand {\FALSE}{\mbox{\bf false\ }}
\newcommand {\FORALL}{\mbox{\bf for all\ }}
\newcommand {\DOWNTO}{\mbox{\bf down to\ }}

% Theorem-type environments
% \theoremstyle{break} 
% \theoremheaderfont{\scshape}
% \theorembodyfont{\slshape}
% \newtheorem{Thm}{Theorem}[section]
% \newtheorem{Lem}[Thm]{Lemma}
% \newtheorem{Cor}[Thm]{Corollary}
% \newtheorem{Prop}[Thm]{Proposition}
% % \theoremstyle{plain} 
% % \theorembodyfont{\rmfamily} 
% \newtheorem{Ex}[Thm]{Exercise}
% \newtheorem{Exa}[Thm]{Example}
% \newtheorem{Rem}[Thm]{Remark}
% % \theorembodyfont{\itshape}
% \newtheorem{Def}[Thm]{Definition}
% \newtheorem{Conj}[Thm]{Conjecture}
% \newtheorem{Obs}[Thm]{Observation}
% \newtheorem{Ques}[Thm]{Question}
%\newenvironment{proof}{\noindent {\sc Proof:}}{$\Box$ \medskip} 
\newenvironment{problems} % Definition of problems
 {\renewcommand{\labelenumi}{\S\theenumi}
	\begin{enumerate}}{\end{enumerate}}


%%%%%%%%%%%%%%%%% Proof Environments

\def\FullBox{\hbox{\vrule width 6pt height 6pt depth 0pt}}
%
%\def\qed{\ifmmode\qquad\FullBox\else{\unskip\nobreak\hfil
%\penalty50\hskip1em\null\nobreak\hfil\FullBox
%\parfillskip=0pt\finalhyphendemerits=0\endgraf}\fi}

\def\qedsketch{\ifmmode\Box\else{\unskip\nobreak\hfil
\penalty50\hskip1em\null\nobreak\hfil$\Box$
\parfillskip=0pt\finalhyphendemerits=0\endgraf}\fi}

%\newenvironment{proof}{\begin{trivlist} \item {\bf Proof:~~}}
 %  {\qed\end{trivlist}}

\newenvironment{proofsketch}{\begin{trivlist} \item {\bf
Proof Sketch:~~}}
  {\qedsketch\end{trivlist}}

\newenvironment{proofof}[1]{\begin{trivlist} \item {\bf Proof
#1:~~}}
  {\qed\end{trivlist}}

\newenvironment{claimproof}{\begin{quotation} \noindent
{\bf Proof of claim:~~}}{\qedsketch\end{quotation}}


%%%%%%%%%%%%%%%%%%%%%%%%%%%%%%%%%%%%%%%%%%%%%%%%%%%%%%%%%%%%%%%%%%%%%%%%%%%
%%%%%%%%%%%%%%%%%%%%%%%%%%%%%%%%%%%%%%%%%%%%%%%%%%%%%%%%%%%%%%%%%%%%%%%%%%%




\newlength{\tpush}
\setlength{\tpush}{2\headheight}
\addtolength{\tpush}{\headsep}

\newcommand{\handout}[5]{
   \noindent
   \begin{center}
   \framebox{ \vbox{ \hbox to \textwidth { {\bf \coursenum\ :\  \coursename} \hfill #5 }
       \vspace{3mm}
       \hbox to \textwidth { {\Large \hfill #2  \hfill} }
       \vspace{1mm}
       \hbox to \textwidth { {\it #3 \hfill #4} }
     }
   }
   \end{center}
   \vspace*{4mm}
   \newcommand{\lecturenum}{#1}
   \addcontentsline{toc}{chapter}{Lecture #1 -- #2}
}

\newcommand{\lecturetitle}[4]{\handout{#1}{#2}{Lecturer: \courseprof
  }{Scribe: #3}{Lecture #1 : #4}}
\newcommand{\guestlecturetitle}[5]{\handout{#1}{#2}{Lecturer:
    #4}{Scribe: #3}{Lecture #1 - #5}}


%%%%%%%%%%%%%%%%%%%%%%%%%%%%%%%%%%%%%%%%%%%%%%%%%%%%%%%%%
%%% Commands to include figures


%% PSfigure

\newcommand{\PSfigure}[3]{\begin{figure}[t] 
  \centerline{\vbox to #2 {\vfil \psfig{figure=#1.eps,height=#2} }} 
  \caption{#3}\label{#1} 
  \end{figure}} 
\newcommand{\twoPSfigures}[5]{\begin{figure*}[t]
  \centerline{%
    \hfil
    \begin{tabular}{c}
        \vbox to #3 {\vfil\psfig{figure=#1.eps,height=#3}} \\ (a)
    \end{tabular}
    \hfil\hfil\hfil
    \begin{tabular}{c}
        \vbox to #3 {\vfil\psfig{figure=#2.eps,height=#3}} \\ (b)
    \end{tabular}
    \hfil}
  \caption{#4}
  \label{#5}
%  \sublabel{#1}{(a)}
%  \sublabel{#2}{(b)}
  \end{figure*}}

\newcounter{fignum}

% fig
%command to insert figure. usage \fig{name}{h}{caption}
%where name.eps is the postscript file and h is the height in inches
%The figure is can be referred to using \ref{name}
\newcommand{\fig}[3]{%
\begin{minipage}{\textwidth}
\centering\epsfig{file=#1.eps,height=#2}
\caption{#3} \label{#1}
\end{minipage}
}%


% ffigure
% Usage: \ffigure{name of file}{height}{caption}{label}
\newcommand{\ffigure}[4]{\begin{figure} 
  \centerline{\vbox to #2 {\hfil \psfig{figure=#1.eps,height=#2} }} 
  \caption{#3}\label{#4} 
  \end{figure}} 

% ffigureh
% Usage: \ffigureh{name of file}{height}{caption}{label}
\newcommand{\ffigureh}[4]{\begin{figure}[!h] 
  \centerline{\vbox to #2 {\vfil \psfig{figure=#1.eps,height=#2} }} 
  \caption{#3}\label{#4} 
  \end{figure}} 


% {{{ draftbox }}}
\ifnum\showdraftbox=1
\newcommand{\draftbox}{\begin{center}
  \fbox{%
    \begin{minipage}{2in}%
      \begin{center}%
%        \begin{Large}%
          \large\textsc{Working Draft}\\%
%        \end{Large}\\
        Please do not distribute%
      \end{center}%
    \end{minipage}%
  }%
\end{center}
\vspace{0.2cm}}
\else
\newcommand{\draftbox}{}
\fi


%% Complexity classes
\newcommand\p{\mbox{\bf P}\xspace}
\newcommand\np{\mbox{\bf NP}\xspace}
\newcommand\cnp{\mbox{\bf coNP}\xspace}
\newcommand\sigmatwo{\mbox{\bf $\Sigma_2$}\xspace}
\newcommand\ppoly{\mbox{\bf $\p_{\bf /poly}$}\xspace}
\newcommand\sigmathree{\mbox{\bf $\Sigma_3$}\xspace}
\newcommand\pitwo{\mbox{\bf $\Pi_2$}\xspace}
\newcommand\rp{\mbox{\bf RP}\xspace}
\newcommand\zpp{\mbox{\bf ZPP}\xspace}
\newcommand\bpp{\mbox{\bf BPP}\xspace}
\newcommand\ph{\mbox{\bf PH}\xspace}
\newcommand\pspace{\mbox{\bf PSPACE}\xspace}
\newcommand\npspace{\mbox{\bf NPSPACE}\xspace}
\newcommand\dl{\mbox{\bf L}\xspace}
\newcommand\ma{\mbox{\bf MA}\xspace}
\newcommand\am{\mbox{\bf AM}\xspace}
\newcommand\nl{\mbox{\bf NL}\xspace}
\newcommand\conl{\mbox{\bf coNL}\xspace}
\newcommand\sharpp{\mbox{\#{\bf P}}\xspace}
\newcommand\parityp{\mbox{$\oplus$ {\bf P}}\xspace}
\newcommand\ip{\mbox{\bf IP}\xspace}
\newcommand\pcp{\mbox{\bf PCP}}
\newcommand\dtime{\mbox{\bf DTIME}}
\newcommand\ntime{\mbox{\bf NTIME}}
\newcommand\dspace{\mbox{\bf SPACE}\xspace}
\newcommand\nspace{\mbox{\bf NSPACE}\xspace}
\newcommand\cnspace{\mbox{\bf coNSPACE}\xspace}
\newcommand\exptime{\mbox{\bf EXP}\xspace}
\newcommand\nexptime{\mbox{\bf NEXP}\xspace}
\newcommand\genclass{\mbox{$\cal C$}\xspace}
\newcommand\cogenclass{\mbox{\bf co$\cal C$}\xspace}
\newcommand\size{\mbox{\bf SIZE}\xspace}
\newcommand\sig{\mathbf \Sigma}
\newcommand\pip{\mathbf \Pi}

%%Computational problems
\newcommand\sat{\mbox{SAT}\xspace}
\newcommand\tsat{\mbox{3SAT}\xspace}
\newcommand\tqbf{\mbox{TQBF}\xspace}

\allowdisplaybreaks

\usepackage{tikz}

\usepackage[
    backend=biber,
% giveninits=true,
% natbib=true,
    style=alphabetic,
    url=false, 
 %   doi=true,
    hyperref,
    backref=true,
    backrefstyle=none,
    maxbibnames=10,
    sortcites
]{biblatex}
\addbibresource{papers.bib}

%%%%%%%%% Authornotes
\newcommand{\Snote}{\Authornote{S}}

\newenvironment{tight_enumerate}{
\begin{enumerate}
 \setlength{\itemsep}{2pt}
 \setlength{\parskip}{1pt}
}{\end{enumerate}}
\newenvironment{tight_itemize}{
\begin{itemize}
 \setlength{\itemsep}{2pt}
 \setlength{\parskip}{1pt}
}{\end{itemize}}



\addbibresource{refs.bib}

%%%%%%%%%%%%%%%%%%%%%%%%%

\begin{document}

\newcommand{\coursenum}{{CSC 2421H}}
\newcommand{\coursename}{{Graphs, Matrices, and Optimization}}
\newcommand{\courseprof}{Sushant Sachdeva}

\lecturetitle{7}{Concentration Bounds}{Fengwei Sun}{29 Oct 2018}

\section{Scalar Chernoff Bound}

\begin{definition}
Let $X_1, \dots, X_t$ be independent random variables such that 

\[
	0 \le X_i \le R, \mathbb{E} \sum_{i} X_i = \sum_{i} \mathbb{E} X_i = \mu
\]

Then for all $0<\epsilon<1$, we have

\[
	P[\sum_{i} X_i \ge (1+\epsilon)\mu] \le e^{-\frac{\epsilon^2 \mu}{3R}}, P[\sum_{i} X_i \le (1-\epsilon)\mu] \le e^{-\frac{\epsilon^2 \mu}{2R}}
\]

\end{definition}

\paragraph{Example:} 

Suppose we conduct $t$ independent tosses of a fair coin. Let $X_i=\begin{cases} 1 \quad if \; heads \\ 0 \qquad o/w \end{cases}$. Then the number of heads in this trial is $\sum_{i=1}^{t} X_i$, and $\mathbb{E}(\# \; of \; heads)=\mathbb{E}\sum_{i=1}^{t} X_i = \frac{t}{2}$.\\

To obtain a good estimate of the probability that we see at least 600 heads out of 1000 tosses, we can apply the Chernoff bound with the parameters $\epsilon=0.2, R=1, \mu=500$, and get

\[
	P(at \; least \; 600 \; heads \; out \; of \; 1000 \; tosses) = P(\sum_{i=1}^{1000} X_i \ge (1+\epsilon) 500) \le e^{-\frac{0.2^2 * 500}{3}}= e^{-\frac{20}{3}} \approx e^{-7}
\]

\paragraph{Question:}

You can a coin with a bias in $\{\frac{1}{2}+\alpha, \frac{1}{2}-\alpha\}$. How many tosses do you need to decide which bias with probability of at least $1-\delta$?

\paragraph{Algorithm:}

\begin{enumerate}
	\item Toss the coin $t$ times independently;
	\item If there are at least $\frac{t}{2}$ heads, output $\frac{1}{2}+\alpha$; otherwise output $\frac{1}{2}-\alpha$.
\end{enumerate}

We would like to bound $P(failure) \le \delta$. \\

\paragraph{Case 1:}
The coin has a bias $\frac{1}{2}+\alpha$. Then $P(failure) = P(\sum X_i \le \frac{t}{2})$. \\

To apply Chernoff, we find that $R=1, \mu=t(\frac{1}{2}+\alpha)$. Since we want $(1-\epsilon)\mu = \frac{t}{2}$, we have $\epsilon = 1-\frac{t}{2\mu} = 1-\frac{1}{2\alpha}$. Therefore,
\[
	P(\sum X_i \le \frac{t}{2}) = P(\sum X_i \le (1-\epsilon)\mu) \le e^{-\frac{\epsilon^2\mu}{2}} \le \delta
\]

Since $\epsilon^2\mu \approx \Theta(t \alpha^2)$, we have 
\[
	t \ge \frac{\Theta(1)}{\alpha^2}\log \frac{1}{\delta}
\]

\paragraph{Case 2:}

The coin has a bias $\frac{1}{2}-\alpha$. Then $P(failure) = P(\sum X_i \ge \frac{t}{2})$. \\

To apply Chernoff, we find that $R=1, \mu=t(\frac{1}{2}-\alpha)$. Since we want $(1+\epsilon)\mu = \frac{t}{2}$, we have $\epsilon = \frac{t}{2\mu}-1 = \frac{1}{2\alpha}-1$. Therefore,
\[
	P(\sum X_i \ge \frac{t}{2}) = P(\sum X_i \ge (1+\epsilon)\mu) \le e^{-\frac{\epsilon^2\mu}{3}} \le \delta
\]

Since $\epsilon^2\mu \approx \Theta(t \alpha^2)$, we have 
\[
	t \ge \frac{\Theta(1)}{\alpha^2}\log \frac{1}{\delta}
\]

Both cases indicate that the $t$ is bounded by the logarithm of $\frac{1}{\delta}$, which means that $t$ would be relatively small even for very small $\delta$.

\section{Matrix Chernoff Bound}

\begin{definition}

Let $X_1, \dots, X_t \in \mathbb{R}^{d \times d}$ be symmetric independent random variables such that

\[
	0 \preceq X_i \preceq RI, \; \mu_{min} I \preceq \mathbb{E}\sum X_i \preceq \mu_{max} I
\]

Then we have

\[
	P[\lambda_{max}(\sum X_i) \ge (1+\epsilon) \mu_{max}] \le d e^{-\frac{\epsilon^2 \mu_{max}}{3R}},
\]

\[
	P[\lambda_{min}(\sum X_i) \le (1-\epsilon) \mu_{min}] \le d e^{-\frac{\epsilon^2 \mu_{min}}{2R}}
\]
	
\end{definition}

\paragraph{Note:}

\begin{enumerate}
	\item The condition that $0 \preceq X_i \preceq RI$ is equivalent to $||X_i|| \le R$, or $\lambda_{max}(X_i) \le R$, where $||A|| = \max_{x \ne 0} \frac{||Ax||}{||x||}$, and when $A$ is symmetric, $||A||=max\{\lambda_{max}, -\lambda_{min}\}$. 
	\item $\mu_{min}=\lambda_{min}(\mathbb{E}\sum X_i), \mu_{max}=\lambda_{max}(\mathbb{E}\sum X_i)$.
\end{enumerate}

\paragraph{Example:}

(Construction of random expander graphs) Suppose we would like to generate an expander graph with $n$ vertices (assuming $n$ is even).

Define a matching as a graph where $d_v=1$ for all vertices $v$. Let $H=\frac{1}{t}$(union of $t$ independent perfect matching). Notice that for all vertices $u$ in the graph $H$, $d_u = 1$. 

Using the matrix Chernoff bound, we can show that $H$ is an expander.\\

The Laplacian of $H$ is $L_H = \sum_{i} \frac{1}{t} L_i$, where $L_i$ is the Laplacian of the $i^{th}$ matching.

Let $X_i = \frac{1}{t} L_i$. We know that $X_i \succeq 0$ and $\lambda_{max}(X_i) = \frac{1}{t} \lambda_{max}(L_i)=\frac{2}{t}$.

Also, if we look at a specific vertex $u$ in each matching, it is connected to all other vertices with equal probablity $\frac{1}{n-1}$. This indicates that $\mathbb{E}L_H = \mathbb{E}L_1 = \frac{1}{n-1} L_{K_n}$. \\

Before we apply Chernoff bound on $X_i's$, one issue we notice is that $\lambda_{min}(X_i)=0$.

To fix that, we let $X_i = \frac{1}{t} L_i + \frac{1}{t(n-1)} \mathbbm{1}\mathbbm{1}^{\top}$. Now we have $\mathbb{E}\sum X_i = \frac{1}{n-1} L_{K_n} + \frac{1}{n-1} \mathbbm{1}\mathbbm{1}^{\top} = \frac{n}{n-1} I_n$, and thus $\mu_{max}=\mu_{min}=\frac{n}{n-1}$.\\

We can also show that, $\lambda_{max}(X_i) \le \frac{2}{t}$ after the change of variable.

Let $y = \hat{y} + c \frac{\mathbbm{1}}{\sqrt{n}}$ where $\hat{y}^{\top}\mathbbm{1}=0$. Then

\[
	y^{\top} X_i y = \hat{y}^{\top} (\frac{1}{t} L_i) \hat{y} + \frac{c^2n}{t(n-1)} \le (\frac{2}{t})\hat{y}^{\top}\hat{y} + \frac{n}{(n-1)t} c^2 \le \frac{2}{t} (\hat{y}^{\top}\hat{y}+c^2) \le \frac{2}{t} ||y||^2 = \frac{2}{t}
\]

\hfill \break

Now we apply the Chernoff bound, and get the following:

\[
	P[\lambda_{max}(\sum X_i) \ge (1+\epsilon) \frac{n}{n-1}] \le n e^\frac{-\epsilon^2 \frac{n}{n-1}}{3 \cdot \frac{2}{t}} = n e^{-\frac{\epsilon^2 t}{6}(\frac{n}{n-1})}
\]

If we pick $t \ge \frac{12}{\epsilon^2} \log n$, we have $P[\lambda_{max}(\sum X_i) \ge (1+\epsilon)\frac{n}{n-1}] \le n \cdot \frac{1}{n^2} = \frac{1}{n}$. \\

Similarly, we have $P[\lambda_{min}(\sum X_i) \le (1-\epsilon) \frac{n}{n-1}] \le \frac{1}{n}$. \\

Therefore, we can conclude that, with probability of at least $1 - \frac{2}{n}$,

\[
	\lambda_{max}(\sum X_i) \le (1+\epsilon)\frac{n}{n-1}, \lambda_{min}(\sum X_i) \ge (1-\epsilon)\frac{n}{n-1}
\]

or

\[
	(1-\epsilon)\frac{n}{n-1} I \preceq \sum X_i \preceq (1+\epsilon)\frac{n}{n-1}
\]

\hfill \break

To see that $H$ is a good approximation of a complete graph, let $ \Pi = I - \frac{1}{n} \mathbbm{1}\mathbbm{1}^{\top} = \frac{1}{n} L_{K_n}$. Notice that $\Pi^2 = \Pi$.

Consider $\Pi^{\top} (\sum X_i) \Pi$. We have 

\[
	(1-\epsilon)\frac{n}{n-1} \frac{1}{n} L_{K_n} \preceq \Pi^{\top} (\sum X_i) \Pi \preceq (1+\epsilon)\frac{n}{n-1} \frac{1}{n} L_{K_n}
\]

Since

\[
	\Pi^{\top} (\sum X_i) \Pi = \Pi^{\top} (L_H + \frac{1}{n-1} \mathbbm{1}\mathbbm{1}^{\top})(I - \frac{1}{n} \mathbbm{1}\mathbbm{1}^{\top}) 
\]

\[
	= \Pi^{\top} (L_H + \frac{1}{n-1} \mathbbm{1}\mathbbm{1}^{\top} - \frac{n}{n(n-1)} \mathbbm{1}\mathbbm{1}^{\top}) = \Pi^{\top} L_H = L_H
\]

Therefore,

\[
	(1-\epsilon)\frac{1}{n-1} L_{K_n} \preceq L_H \preceq (1+\epsilon)\frac{1}{n-1} L_{K_n}
\]

\hfill \break

Now we get an $\epsilon$-expander $H$ with $t \cdot \frac{n}{2} = \Theta (\frac{n\log n}{\epsilon^2})$ edges, and $L_H \approx_\epsilon L_{K_n}$.\\

\hfill \break

In general, we would like to write the Chernoff bound as the following: \\

With probability of at least $1 - 2de^{-\frac{\epsilon^2 \mu_{min}}{2R}}$, $(1-\epsilon) \mu_{min} I \preceq \sum X_i \preceq (1+\epsilon)\mu_{max} I$. \\


\begin{definition}
	
	$H=(V, E')$ is an $\epsilon$-spectral sparsifier of $G=(V, E)$ if $\frac{1}{1+\epsilon} L_G \preceq L_H \preceq (1+\epsilon) L_G$, denoted as $L_H \approx_\epsilon L_G$.
	
	Equivalently, $\forall x \in \mathbb{R}^V, \frac{1}{1+\epsilon} x^{\top} L_G x \le x^{\top} L_H x \le (1+\epsilon) x^{\top} L_G x$.
	
\end{definition}

\paragraph{Note:}

Let $x = \mathbbm{1}_S$ where $S \subset V$. Then $x^{\top} L_H x = \sum_{(u,v) \in E} w(u, v) (x(u)-x(v))^2 = |E(S, \bar{S})|$.


\paragraph{Theorem:}

For all $G=(V, E)$, there exists $H=(V, E')$ such that $L_H \approx_\epsilon L_G$ and $|E'| \le \Theta(\frac{n\log n}{\epsilon^2})$



\printbibliography
\end{document}


%%% Local Variables:
%%% mode: latex
%%% TeX-master: t
%%% End:
